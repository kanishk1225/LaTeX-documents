\documentclass[11pt]{article}
\usepackage{comment}
\usepackage{graphicx}
\usepackage[top=.25 in,bottom=.25 in,right=1 in,left=1 in]{geometry}
\usepackage{amssymb}
\usepackage{amsfonts}
\usepackage{amsmath}
\usepackage[utf8]{inputenc}
\usepackage{fancyhdr}


%%\pagestyle{fancy}
%%\lhead{}
%%\chead{}
%%\rhead{}
%%\cfoot{\thepage}
%%\lfoot{}
%%\rfoot{}
%%\renewcommand{\headrulewidth}{0.4pt}
%%\renewcommand{\footrulewidth}{0.4pt}


\begin{document}

\begin{large}
\title{Program v/s Project}
\author{}
\date{}
\maketitle
\begin{flushleft}
\begin{tabular}{|p{4cm}|p{6cm}|p{6cm}|}
\hline
	& 	Project	&	Program\\
\hline

Objectives	&	Outputs – tangible; 

relatively easy to describe, 

define and measure; tending towards objective.	

& Outcomes – often  intangible; 

difficult to quantify;

 benefits often based on changes to organizational culture and behaviors;
 
  introducing new capabilities into the organization; 
  
  tending towards subjective. \\

\hline

Scope	& Strictly limited; tightly defined; not likely to be subject to material change during the life of the project.	& Not tightly defined or bounded; likely to change during the life cycle of the program. \\

\hline

Duration	& Relatively short term; typically three to six months.	&Relatively long term typically eighteen months to three years. \\

\hline

Risk profile	& Project risk is relatively easy to identify and manage. The project failure would result in relatively limited impact on the organization relative to program risk.	& Program risk is more complex and potentially the impact on the organization if a risk materializes will be greater relative to project risk. Programme failure could result in material financial, reputational or operational loss.\\

\hline

Nature of the problem	& Clearly defined.	& Ill-defined; often disagreement between key stakeholders on the nature and definition of the problem.\\

\hline

Nature of the solution	& A relatively limited number of potential solutions. 	& A significant number of potential solutions with often with disagreement between stakeholders as to the preferred solution.\\

\hline
 
Stakeholders	& A relatively limited number of stakeholders. 	& A significant number of diverse stakeholders; probable disagreement between them as to the definition of the problem \& the preferred solution. \\

\hline

Relationship to environment	&
Environment within which the project takes place is understood and relatively stable.	&
Environment is dynamic; and programme objectives need to be managed in the context of the changing environment within which the organization operates. \\

Resources	&
Resources to deliver the project can be reasonably estimated in advance. 	&
Resources are constrained and limited; there is competition for resources between projects.\\


\hline
\end{tabular}
\end{flushleft}

\leavevmode\thispagestyle{empty}\newpage
\
\
\
\
\

\begin{flushleft}


\begin{Large}
\textbf{GROUP MEMBERS}
\end{Large}
	\begin{enumerate}
\item Kanishk Goyal (15114035)
		
\item Ujjawal (15114074)
		
\item Siraz Shaikh (15114065)
		
\item Sunil Yadav (14114064)
	
		
	\end{enumerate}
	
\end{flushleft}
\end{large}

\end{document}
