\documentclass[18 pt]{article}
\usepackage{comment}
\usepackage{graphicx}
\usepackage[top=1 in,bottom=1 in,right=1 in,left=1 in]{geometry}
\usepackage{amssymb}
\usepackage{amsfonts}
\usepackage{amsmath}
\usepackage[utf8]{inputenc}
\usepackage{fancyhdr}


\pagestyle{fancy}
\lhead{}
\chead{}
\rhead{}
\cfoot{\thepage}
\lfoot{}
\rfoot{}
\renewcommand{\headrulewidth}{0.4pt}
\renewcommand{\footrulewidth}{0.4pt}


\begin{document}
\begin{large}

\title{\textbf{Project Proposal}}
\author{}
\date{}
\maketitle



\section*{Section 1: Project Proposal}	



\subsection*{Project Title: } Image data extraction
\subsection*{Team: }

\begin{enumerate}
\item Kanishk Goyal \	(15114035)
		
\item Siraz Shaikh 	\	(15114065)
		
\item Ujjawal 	\	\	(15114074)
		
\item Sunil Yadav 	\	(14114064)
\end{enumerate}

\subsection*{Proposal:}
The product aims to solve the problem of reading text from the images. As we all are aware of the fact that in today’s scenario how important data is, how important is the searching of relevant information is. Therefore, we are presenting a software which converts the text in the image a into text document and then operations can be performed on it.


This product will not only make the life easy by providing different information easily accessible to the customer but also enables him to do various kind of operations on it. For example, researchers have large data in the form of images such as, graphs, hand written documents, and their field data. This product will help them in creating a well documented reports. Students have large number of notes in the form of images. Thus, this product will help students in editing and managing their notes.

I would like to show the diversity of customers it can approach. Our product can not only help in the field of education but can also help in industries, forensic laboratory, scientists, and for all those who wanted to access information hidden in an image. 


This could revolutionize the way search engine works. Today the result of our web search depends on the number of other web pages referring to it and the count of the key word of our search in that particular web page. Give it a thought, with our approach to see the images not only as a source of pictorial data but also as a collection of data with text and logos in it. 



\section*{Section 2: Feasibility Report}
\subsection*{Customer:}
Our project aim to the large section of the society:
\begin{enumerate}
\item	\textbf{Students:} large number of notes are in the form of images. So, this will help them in editing and managing their notes.
\item	\textbf{Researchers:} they have large data in the form of images such as, graphs, hand written documents, and their field data. This will help them in creating a well documented reports.
\item	\textbf{Crime Investigation Department:} this department has large amount of data in the form of images as evidence and clues which when documented in their database will help them in connecting the dots.
\item	\textbf{Industries with image data processing:} this section of the society already deals with the image processing and major part of image processing is collecting the information from image and this information are in the form of figures, text, design, specified information according to the aim of processing, and logos. This will help them in every aspect of it in later versions.
\end{enumerate}

\subsection*{Customer interaction:} 
There will be an email address specifically for customer support, feedback, suggestion.

\subsection*{Interaction between the team members:}
We have a Gitter group for freelancer.
We have a Github closed repository for true contributors and team members.
We have a complete documentation of deadlines and task distribution.

\subsection*{Preliminary requirements:}
There is a requirement of dataset of different fonts of text for recognising different style of text.

\subsection*{Suggested deliverables:}
End product will be the android application which will take the image as the input and create text document for the text in image and include a search option. In the later updates of the software will include logos detection and storing them in the document with their description (or just name).

\subsection*{Process to be followed:}
The development process followed is iterative waterfall model :
\begin{center}
\includegraphics{Waterfall.png}
\end{center}

\subsection*{Outline Plan and Principle activities:}
The outline of steps to be followed in development phase of software development life cycle are:
\begin{enumerate}
\item	Dataset extraction from open database available online.

\item	Building Basic structure of application.

\item	Adding features in the application for image processing.

\item	Adding search option in the application.



\end{enumerate}


\subsection*{Risk Analysis:}
The initial problems in the project:
\begin{enumerate}
\item	There are chances that the upgrade version of software is not developed in time.
\item	There are chances that dataset which we are extracting from the online source might be too large that the system hangs.

\end{enumerate}
The fallback plan is:
\begin{enumerate}
\item	For the issue of time taken in development there will be an proper distribution of work based on team members ability.
\item	For large dataset, there will be a subset of the dataset used to launch the software in market.
\end{enumerate}

\subsection*{Technical Requirements:}
There are numerous technical requirements for the project:
\begin{enumerate}
\item	Android Studio (for android app development)
\item	Tesorflow Library (for image processing)
\item	APIs from open source (for dataset)
\end{enumerate}




\end{large}
\end{document}