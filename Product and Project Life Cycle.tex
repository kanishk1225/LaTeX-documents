\documentclass[18 pt]{article}
\usepackage{comment}
\usepackage{graphicx}
\usepackage[top=1 in,bottom=1 in,right=1 in,left=1 in]{geometry}
\usepackage{amssymb}
\usepackage{amsfonts}
\usepackage{amsmath}
\usepackage[utf8]{inputenc}
\usepackage{fancyhdr}


\pagestyle{fancy}
\lhead{}
\chead{}
\rhead{}
\cfoot{\thepage}
\lfoot{}
\rfoot{}
\renewcommand{\headrulewidth}{0.4pt}
\renewcommand{\footrulewidth}{0.4pt}


\begin{document}
\begin{large}

\title{Product and Project Life Cycle}
\author{}
\date{}
\maketitle


\tableofcontents

\leavevmode\thispagestyle{empty}\newpage

\section{Product Life Cycle Stages}	
 As consumers, we buy millions of products every year. And just like us, these products have a life cycle. Older, long-established products eventually become less popular, while in contrast, the demand for new, more modern goods usually increases quite rapidly after they are launched.
 
 
Because most companies understand the different product life cycle stages, and that the products they sell all have a limited lifespan, the majority of them will invest heavily in new product development in order to make sure that their businesses continue to grow.
\begin{center}
\includegraphics{productLifeCycleStages.jpg}
\end{center}
\subsection{Product Life Cycle Stages Explained}

The product life cycle has 4 very clearly defined stages, each with its own characteristics that mean different things for business that are trying to manage the life cycle of their particular products.

\subsubsection{Introduction Stage } This stage of the cycle could be the most expensive for a company launching a new product. The size of the market for the product is small, which means sales are low, although they will be increasing. On the other hand, the cost of things like research and development, consumer testing, and the marketing needed to launch the product can be very high, especially if it’s a competitive sector.


\subsubsection{Growth Stage } The growth stage is typically characterized by a strong growth in sales and profits, and because the company can start to benefit from economies of scale in production, the profit margins, as well as the overall amount of profit, will increase. This makes it possible for businesses to invest more money in the promotional activity to maximize the potential of this growth stage.

\subsubsection{Maturity Stage } During the maturity stage, the product is established and the aim for the manufacturer is now to maintain the market share they have built up. This is probably the most competitive time for most products and businesses need to invest wisely in any marketing they undertake. They also need to consider any product modifications or improvements to the production process which might give them a competitive advantage.


\subsubsection{Decline Stage } Eventually, the market for a product will start to shrink, and this is what’s known as the decline stage. This shrinkage could be due to the market becoming saturated (i.e. all the customers who will buy the product have already purchased it), or because the consumers are switching to a different type of product. While this decline may be inevitable, it may still be possible for companies to make some profit by switching to less-expensive production methods and cheaper markets.


\section{The Project Life Cycle (Phases)}

The project manager and project team have one shared goal: to carry out the work of the project for the purpose of meeting the project’s objectives. Every project has a beginning, a middle period during which activities move the project toward completion, and an ending (either successful or unsuccessful). A standard project typically has the following four major phases (each with its own agenda of tasks and issues): initiation, planning, implementation, and closure. Taken together, these phases represent the path a project takes from the beginning to its end and are generally referred to as the project “life cycle.”
\begin{center}
\includegraphics{projectLifeCycleStages.jpg}
\end{center}
\subsection{Initiation Phase}
During the first of these phases, the initiation phase, the project objective or need is identified; this can be a business problem or opportunity. An appropriate response to the need is documented in a business case with recommended solution options. A feasibility study is conducted to investigate whether each option addresses the project objective and a final recommended solution is determined. Issues of feasibility (“can we do the project?”) and justification (“should we do the project?”) are addressed.


Once the recommended solution is approved, a project is initiated to deliver the approved solution and a project manager is appointed. The major deliverables and the participating work groups are identified, and the project team begins to take shape. Approval is then sought by the project manager to move onto the detailed planning phase.
\subsection{Planning Phase}
The next phase, the planning phase, is where the project solution is further developed in as much detail as possible and the steps necessary to meet the project’s objective are planned. In this step, the team identifies all of the work to be done. The project’s tasks and resource requirements are identified, along with the strategy for producing them. This is also referred to as “scope management.” A project plan is created outlining the activities, tasks, dependencies, and timeframes. The project manager coordinates the preparation of a project budget by providing cost estimates for the labor, equipment, and materials costs. The budget is used to monitor and control cost expenditures during project implementation.


Once the project team has identified the work, prepared the schedule, and estimated the costs, the three fundamental components of the planning process are complete. This is an excellent time to identify and try to deal with anything that might pose a threat to the successful completion of the project. This is called risk management. In risk management, “high-threat” potential problems are identified along with the action that is to be taken on each high-threat potential problem, either to reduce the probability that the problem will occur or to reduce the impact on the project if it does occur. This is also a good time to identify all project stakeholders and establish a communication plan describing the information needed and the delivery method to be used to keep the stakeholders informed.


Finally, you will want to document a quality plan, providing quality targets, assurance, and control measures, along with an acceptance plan, listing the criteria to be met to gain customer acceptance. At this point, the project would have been planned in detail and is ready to be executed.


\subsection{Implementation (Execution) Phase}
During the third phase, the implementation phase, the project plan is put into motion and the work of the project is performed. It is important to maintain control and communicate as needed during implementation. Progress is continuously monitored and appropriate adjustments are made and recorded as variances from the original plan. In any project, a project manager spends most of the time in this step. During project implementation, people are carrying out the tasks, and progress information is being reported through regular team meetings. The project manager uses this information to maintain control over the direction of the project by comparing the progress reports with the project plan to measure the performance of the project activities and take corrective action as needed. The first course of action should always be to bring the project back on course (i.e., to return it to the original plan). If that cannot happen, the team should record variations from the original plan and record and publish modifications to the plan. Throughout this step, project sponsors and other key stakeholders should be kept informed of the project’s status according to the agreed-on frequency and format of communication. The plan should be updated and published on a regular basis.


Status reports should always emphasize the anticipated end point in terms of cost, schedule, and quality of deliverables. Each project deliverable produced should be reviewed for quality and measured against the acceptance criteria. Once all of the deliverables have been produced and the customer has accepted the final solution, the project is ready for closure.
\subsection{Closing Phase}
During the final closure, or completion phase, the emphasis is on releasing the final deliverables to the customer, handing over project documentation to the business, terminating supplier contracts, releasing project resources, and communicating the closure of the project to all stakeholders. The last remaining step is to conduct lessons-learned studies to examine what went well and what didn’t. Through this type of analysis, the wisdom of experience is transferred back to the project organization, which will help future project teams.


\begin{flushleft}


\begin{Large}
\textbf{GROUP MEMBERS}
\end{Large}
	\begin{enumerate}
\item Kanishk Goyal (15114035)
		
\item Ujjawal (15114074)
		
\item Siraz Shaikh (15114065)
		
\item Sunil Yadav (14114064)
	
		
	\end{enumerate}
	
\end{flushleft}
\end{large}
\end{document}